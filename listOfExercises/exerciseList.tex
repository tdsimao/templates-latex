\documentclass[12pt]{article}
\usepackage[brazil]{babel} %  portuguese of Brazil
\usepackage[latin1]{inputenc}
\usepackage[T1]{fontenc}
\usepackage{fullpage} %remove margin

\usepackage{framed} %to draw header

%include commands of exercises and counters


%\newenvironment{exercise}[3][Exerc�cio]{\textbf{#1 #2}  #3  \\} {\vspace{32pt}}

\newcounter{exercise}[section]
\newcounter{subexercise}[exercise]




\newcommand{\exercise}[1]{
    \stepcounter{exercise} %atualiza contador do comando
    \vspace{16pt}\noindent %espa�o anterior
    %\textsf{\textbf{\arabic{exercise}) #1}} %negrito com outra fonte
    \textbf{\arabic{exercise})} #1 %negrito sem mudar fonte
}
\newcommand{\subexercise}[1]{
    \stepcounter{subexercise} %atualiza contador do comando
    \vspace{8pt}    %espa�o
    \noindent
    %\textsf{\textbf{\alph{subexercise}) #1:}} %negrito trocando a fonte
    \textbf{\alph{subexercise})} #1 %negrito sem trocar fonte
} 


\newtheorem{demonstration}{Demontra��o}


\newcounter{step}[demonstration]
\newcommand{\step}[0]{
    \stepcounter{step} %atualiza contador do comando
    \vspace{12pt}\noindent %espa�o anterior
    \textbf{Passo \arabic{step}:}
}


% Template to define actions with parameters, preconditions and effects
\newcommand{\actionParPreEf}[4]{

    \noindent Action( \textbf{#1} (#2),\\
    \indent Pre: #3 \\
    \indent Ef: #4
}

% Template to define actions with parameters, preconditions, positive effects and negative effects
\newcommand{\actionParPreEfEf}[5]{

    \noindent Action( \textbf{#1} (#2),\\
    \indent Pre: #3 \\
    \indent Ef$^+$: #4 \\
    \indent Ef$^-$: #5
}




\newcommand{\Disciplina}{Disciplina}
\newcommand{\Titulo}{Name of the Homework}
\newcommand{\Subtitulo}{Subtitle of the Homework}
\newcommand{\Prof}{Name of the teacher}

\newcommand{\Autor}{Students Name (other information)}
\newcommand{\Data}{\today }
\newcommand{\Local}{Departament-University}


\title{ \Titulo}
\author{\Autor}
\date{\Data}


\begin{document}




\begin{framed}
\noindent
    \begin{minipage}{0.44\textwidth}\begin{flushleft}
        {\footnotesize \Disciplina }
    \end{flushleft}\end{minipage}
    \begin{minipage}{0.54\textwidth}\begin{flushright}
        {\footnotesize  \Local }
    \end{flushright}
    \end{minipage}
\\[1em]
\noindent
    \begin{minipage}{.98\textwidth}\begin{center}
        {\large  \textbf{\Titulo}
        } \\[.1em]
        
        { \footnotesize \Subtitulo \\[.1em]} % subtitle
     
    {\Data}
    \end{center}
    \end{minipage}
\\[1em]
    \begin{minipage}{0.44\textwidth}\begin{flushleft}
        {\footnotesize Prof(a): \Prof }
    \end{flushleft}\end{minipage}
    \begin{minipage}{0.54\textwidth}\begin{flushright}
        {\footnotesize \textbf{\Autor}}
    \end{flushright}
    \end{minipage}
\end{framed} %personalized header
%\maketitle %uncomment for classical title

%Example of exercise
\exercise{About your favorite band:}
    
    %Defining a subexercise
    \subexercise{Name:}\\
        %Students can answer the question here
        AC/DC 
        
    %another subexercise
    \subexercise{Best album:}\\
        Back in Black

    %another subexercise
    \subexercise{Foundation year:}\\
       1973 
        
        
Don't forget to break lines and always give credits to your references

If you want to make a demontration and split it in different steps you can use
the command 
\begin{verbatim}
\step
\end{verbatim}


\begin{demonstration} Super demonstration 
\end{demonstration}

For example:

\step First we must have an equation
\[f(x)=0\]

\step Then we can set a value to our $x$
\[x=1\]

\step Finally in our last step we calculate the result
\[f(1)=0\]
demostra��o

\begin{demonstration} Second Super demonstration demostra��o
\end{demonstration}

For example:

\step First we must have an equation
\[f(x)=0\]

\end{document} 
